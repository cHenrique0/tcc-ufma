\documentclass[a4paper, 12pt]{article}
\usepackage[top=2cm, bottom=2cm, left=2.5cm, right=2.5cm]{geometry}
\usepackage[utf8]{inputenc}
\usepackage{amsmath, amsfonts, amssymb}
\usepackage{float}
\usepackage{graphicx}
\usepackage[brazil]{babel}
\usepackage{indentfirst}

\begin{document}

\begin{center}
	\part*{Resumo}
\end{center}

\begin{center}
	\part*{Abstract}
\end{center}

\section{Introdução}

\textbf{1. Histórico}

\textbf{2. Conceitos de Automação industrial}

\textbf{2.1. Automação e mão de obra}

\textbf{2.2. Automação e controle}

\textbf{2.3. Automação e eletrônica}

\textbf{3. Sistemas de Automação}

\textbf{4. Processos industriais}

\textbf{VANTAGENS DA AUTOMAÇÃO}

\subsection{Justificativa}

Enfatizar a Não parada da linha de produção

\subsection{Objetivo Geral}

Este trabalho tem o propósito de/pretende/visa/ mostrar a importância da 
ultilização de ferramentas de simulação para a programação de máquinas de
automação de processos industriais.

(citar a bancada Festo?)

\subsection{Objetivo Especifico}

\begin{itemize}
  \item Projetar uma linha de produção industrial
  \item Implementar o projeto utilizando CLP a Ladder Logic
  \item Simular a implementação usando o Factory I/O
  \item Utilizar a implementação na planta real (Bancada Festo?)
  \item Verificar ...
\end{itemize}

\section{Referencial Teórico}

	\subsection{Controlador Lógico Programável}
	
	O Controlador Lógico Programável - \textit{CLP} (do inglês, \textit{Programmable Logic Computer - PLC}),
	é uma ferramenta fundamental na industria, pois foi projetado para o uso em um
	ambiente industrial, capaz de resistir às adversidades presentes em uma fábrica,
	tais como água, temperaturas extremas, impactos, sujeira em excesso, dentre outras.
		
	A International Electrotechinical Commission - \textit{IEC}, na norma 61131, define
	o CLP como sendo um sistema eletrônico operando digitalmente, projetado 
	para uso em um ambiente industrial, que usa uma  	memória programável para armazenar
	internamente instruções orientadas para o usuário implementar funções especificas,
	tais como lógica, seqüencial, temporização, contagem e aritmética, para controlar,
	através de entradas e saídas	digitais ou analógicas, vários tipos de máquinas ou 
	processos.
	
	O CLP é amplamente utilizado para o controle de processos industriais, por apresentar
	alguns benefícios como a facilidade de instalação e programação, compatibilidade de rede,
	verificação de defeitos, confiabilidade, além de reduzir muito a quantidade de fios e cabos
	presente nos circuitos de controle a relé, como mostram as figuras X e Y.
	
	\textbf{INSERIR AS FIGURAS CLPControlPanel e ReleControlPanel}
	
	Os CLPs tem algumas outras vantagens em relação aos controles a relé convencionais. Enquanto os relés
	precisam ser instalados para executar uma função específica, e quando o sistema precisa de uma modificação,
	os condutores do relé precisam ser substituídos ou modificados, os CLPs são mais flexíveis ao permitir
	que o usuário apenas crie ou altere a lógica do programa armazenado nele.
	Em controles a relé, o modo como os relés são interconectados regem as relações entre	entradas e saídas,
	em um CLP, o usuário do programa é quem determina estas relações. A figura X ilustra as relações:
	
	\textbf{INSERIR A FIGURA CLPInputOuputRelations}
	
	Os CLPs tem grande capacidade de comunicação. Um CLP pode comunicar-se com outros CLPs ou qualquer outro
	equipamento do computador criando uma rede capaz de realizar funções como supervisão do controle, coleta de dados,
	dispositivos de monitoramento, e permitindo também baixar e transferir programas.
	
		\subsubsection{Partes de um CLP}
		Página 4 - Livro [2]
	
		\subsubsection{Siemens S7-1200}
		DO MANUAL:
		1. Introdução
		1.2 Capacidade de expansão
		3. Software de programação
		3.2 pra pegar imagens
		4. pra pegar imagens do CLP
		6. Device configuration (Imagens tbm)
		A.6 e A.6.3 (Pegar quantidade de portas e etc)
		C. Ordering Information

	\subsection{Ladder Logic}

	\subsection{Grafcet}

	\subsection{SCADA}

	\subsection{Bancada Festo}
	
	\subsection{TIA Portal v?}
	DO manual:
	7. Programming Concepts
	
	\subsection{Factory I/O}
	
	\subsection{Digital Twin}

\section{Metodologia do trabalho}

\section{Cronograma}

\section{Referencias Bibliográficas}

[1] Automação Industrial - Marco Antonio Ribeiro

[2] Controladores logicos programaveis - Frank D. Petruzella

[3] Automação Industrial na Prática - Frank Lamb

[4] Controladores Lógicos Programáveis - Claiton Moro Franchi

[5] A quarta revolução industrial - Klaus Schwab

[6] Automacao Industrial - Leandro Roggia \& Rodrigo Cardozo Fuentes

[7] Introdução à Automação - Mauro Spinola

[8] S7-1200 System Manual - Siemens

[9] IEC 61131

[10] IEC 61131-3

\end{document}
