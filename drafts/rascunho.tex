\documentclass[a4paper, 12pt]{article}
\usepackage[top=2cm, bottom=2cm, left=2.5cm, right=2.5cm]{geometry}
\usepackage[utf8]{inputenc}
\usepackage{amsmath, amsfonts, amssymb}
\usepackage{float}
\usepackage{graphicx}
\usepackage[brazil]{babel}
\usepackage{indentfirst}

\begin{document}

\begin{center}
\part*{Resumo}
\end{center}

\begin{center}
\part*{Abstract}
\end{center}

\section{Introdução}

\textbf{1. Histórico}

``A evolução da automação industrial remete há longos períodos de tempo
na história. Desde a pré-história o homem vem desenvolvendo mecanismos
e invenções com o intuito de reduzir o esforço físico e auxiliar na realização
de atividades. Como exemplo, podem-se citar a roda para movimentação de
cargas e os moinhos movidos por vento ou força animal."[6]

``Entretanto, a automação industrial começou a conquistar destaque na sociedade
no século XVIII, com o início da Revolução Industrial, originada na Inglaterra.
Devido a uma evolução no modo de produção, o homem passou a produzir
mercadorias em maior escala."[6]

``Com o objetivo de aumentar a produtividade, diversas inovações tecnológicas
foram desenvolvidas no período:
\begin{itemize}
	\item Máquinas modernas, capazes de produzir com maior precisão e rapidez
				quando comparadas ao trabalho manual.
	\item Novas fontes energéticas, como o vapor, aplicado a máquinas para 
				substituir a energia hidráulica e/ou muscular.
\end{itemize}
"[6]

``A partir do século XIX, a energia elétrica passou a ser utilizada e a estimular
indústrias como a do aço e a química. Novos processos de produção de aço, que
aumentam a sua resistência e permitem a sua produção em escala industrial,
foram criados. O setor de comunicações passou por avanços significativos
com as invenções do telégrafo e do telefone. O setor de transportes também
progrediu com a expansão das estradas de ferro, locomotivas a vapor e o
crescimento da indústria naval. Outra importante invenção, o motor à explosão,
também ocorreu neste período."[6]

``No século XX, computadores, servomecanismos e controladores programáveis
passaram a fazer parte da automação. Durante este período,
George Boole desenvolveu a álgebra booleana (assunto abordado na Aula 4),
que apresenta os princípios binários, os quais são aplicados nas operações
internas de computadores. Os computadores constituem a base de toda a
tecnologia da automação contemporânea e exemplos de sua aplicação estão
presentes em praticamente todas as áreas do conhecimento."[6]

\textbf{2. Conceitos de Automação industrial}

``Automação é a substituição do trabalho humano ou animal por máquina.
Automação é a operação de máquina ou de sistema automaticamente ou por
controle remoto, com a mínima interferência do operador humano.
Automação é o controle de processos automáticos. Automático significa ter
um mecanismo de atuação própria, que faça uma ação requerida em tempo 
determinado ou em resposta a certas condições."[1]

``O conceito de automação inclui a idéia de usar a potência elétrica ou 
mecânica para acionar algum tipo de máquina. Deve acrescentar à máquina
algum tipo de inteligência para que ela execute sua tarefa de modo mais
eficiente e com vantagens econômicas e de segurança."[1]

``Como vantagens, a máquina: Não reclama, não entra em greve, não pede 
aumento de salário, não precisa de férias, não requer mordomias.
Como nada é perfeito, a máquina tem as seguintes limitações: capacidade 
limitada de tomar decisões, deve ser programada ou ajustada para controlar
sua operação nas condições especificadas, necessita de calibração periódica
para garantir sua exatidão nominal, requer manutenção eventual para assegurar
que sua precisão nominal não se degrade."[1]

\textbf{2.1. Automação e mão de obra}

``Com o advento do circuito integrado(1960) e do microprocessador (1970), a
quantidade de inteligência que pode ser embutida em uma máquina a um custo
razoável se tornou enorme. O número de tarefas complexas que podem ser feitas
automaticamente cresceu várias vezes. Atualmente, pode-se dedicar ao computador
pessoal para fazer tarefas simples e complicadas, de modo econômico."[1]

``A automação pode reduzir a mão de obra empregada, porém ela também e
ainda requer operadores. Em vez de fazer a tarefa diretamente, o operador
controla a máquina que faz a tarefa. Assim, a dona de casa deve aprender a
carregar a máquina de lavar roupa ou louça e deve conhecer suas limitações.
Operar a máquina de lavar roupa pode inicialmente parecer mais difícil que
lavar a roupa diretamente. Do mesmo modo, o operador de uma furadeira 
automática na indústria automobilística deve ser treinado para usar
a máquina com controle numérico que faz o furo realmente. A linha de 
montagem com robôs requer operadores para monitorar o desempenho desses robôs.
Quem tira o dinheiro do caixa eletrônico, deve possuir um cartão apropriado,
decorar uma determinada senha e executar uma série de comandos no teclado 
ou tela de toque."[1]

``Muitas pessoas pensam e temem que a automação significa perda de empregos,
quando pode ocorrer o contrário. De fato, falta de automação coloca muita gente
para trabalhar. Porém, estas empresas não podem competir economicamente com
outras por causa de sua baixa produtividade devida à falta de automação
e por isso elas são forçadas a demitir gente ou mesmo encerrar suas atividades.
Assim, automação pode significar ganho e estabilidade do emprego, por causa do
aumento da produtividade, eficiência e economia."[1]

\textbf{2.2. Automação e controle}

``A automação está intimamente ligada à instrumentação. Os diferentes
instrumentos são usados para realizar a automação.
Historicamente, o primeiro termo usado foi o de controle automático de processo.
Foram usados instrumentos com as funções de medir, transmitir, comparar e
atuar no processo, para se conseguir um produto desejado com pequena ou
nenhuma ajuda humana. Isto é controle automático.
Com o aumento da complexidade dos processos, tamanho das plantas, exigências de
produtividade, segurança e proteção do meio ambiente, além do controle automático
do processo, apareceu a necessidade de monitorar o controle automático."[1]

``Por isso, para o autor, principalmente para a preparação de seus cursos e
divisão de assuntos, tem-se o controle automático aplicado a processo contínuo,
com predominância de medição, controle PID (proporcional, integral e derivativo).
O sistema de controle aplicado é o Sistema Digital de Controle Distribuído (SDCD),
dedicado a grandes plantas ou o controlador single loop, para aplicações
simples e com poucas malhas. Tem-se a automação associada ao controle automático,
para fazer sua monitoração, incluindo as tarefas de alarme e intertravamento.
A automação é também aplicada a processos discretos e de batelada, onde há muita
operação lógica de ligar e desligar e o controle seqüencial.
O sistema de controle aplicado é o Controlador Lógico Programável (CLP)."[1]

\textbf{2.3. Automação e eletrônica}

``Na década de 1970, era clássica a comparação entre as instrumentações
eletrônica e pneumática. Hoje, às vésperas do ano 2000, há a predominância da
eletrônica microprocessada.
Os sensores que medem o valor ou estado de variáveis importantes em um
sistema de controle são as entradas do sistema, mas o coração do sistema é o
controlador eletrônico microprocessado. Muitos sistemas de automação só se
tornaram possíveis por causa dos recentes e grandes avanços na eletrônica. 
Sistemas de controle que não eram práticos por causa de custo há cinco anos 
atrás hoje se tornam obsoletos por causa do rápido avanço da tecnologia."[1]

``A chave do sucesso da automação é o uso da eletrônica microprocessada que
pode fornecer sistemas eletrônicos programáveis. Por exemplo, a indústria
aeronáutica constrói seus aviões comerciais em uma linha de montagem,
mas personaliza o interior da cabine através de simples troca de um programa
de computador. A indústria automobilística usa robôs para soldar pontos e 
fazer furos na estrutura do carro. A posição dos pontos de solda, o diâmetro
e a profundidade dos furos e todas as outras especificações podem ser alteradas
através da simples mudança do programa do computador. Como o programa do 
computador é armazenado em um chip de memória, a alteração de linhas do programa
neste chip pode requerer somente alguns minutos. Mesmo quando se tem que 
reescrever o programa, o tempo e custo envolvidos são muitas vezes menores que o
tempo e custo para alterar as ferramentas."[1]

\textbf{3. Sistemas de Automação}

``A aplicação de automação eletrônica nos processos industriais resultou em
vários tipos de sistemas, que podem ser geralmente classificados como:
\begin{enumerate}
	\item Máquinas com controle numérico
	\item Controlador lógico programável
	\item Sistema automático de armazenagem e recuperação
	\item Robótica
	\item Sistemas flexíveis de manufatura.
\end{enumerate}

\textbf{3.1 Maquinas com controle numerico}

EXPLICAR RAPIDAMENTE

\textbf{3.2 Controlador lógico programável}

O controlador lógico programável é um equipamento eletrônico, digital, 
microprocessado, que pode:
\begin{enumerate}
	\item controlar um processo ou uma máquina
	\item ser programado ou reprogramado rapidamente e quando necessário
	\item ter memória para guardar o programa.
\end{enumerate}
O programa é inserido no controlador através de microcomputador, teclado
numérico portátil ou programador dedicado. O controlador lógico programável
varia na complexidade da operação que eles podem controlar, mas eles podem ser
interfaceados com microcomputador e operados como um DNC, para aumentar
sua flexibilidade. Por outro lado, eles são relativamente baratos, fáceis 
de projetar e instalar.

\textbf{3.3. Sistema de armazenagem e recuperação de dados}

EXPLICAR RAPIDAMENTE

\textbf{3.4. Robótica}

EXPLICAR RAPIDAMENTE

\textbf{3.5. Sistema de manufatura flexível}

EXPLICAR RAPIDAMENTE
"[1]

\textbf{4. Processos industriais}

``Basicamente, a automação industrial pode ser dividida em duas modalidades
quanto aos tipos de processos: processos da manufatura e processos contínuos.
Os processos da manufatura são aqueles em que há grande movimentação
mecânica de partes. O exemplo mais clássico é a indústria automobilística. Na
linha de montagem, há robôs soldadores, esteiras transportadoras e outros
sistemas, como mostra a Figura 1. Nos processos da manufatura, as grandezas
mais comuns são força, velocidade e deslocamento."[6]

\textbf{INSERIR FIGURA 1}

``Ao contrário dos processos da manufatura, os processos contínuos são carac-
terizados pela pouca movimentação mecânica de partes. Uma estação de
tratamento de água, mostrada na Figura 2, é um exemplo. As grandezas
mais comuns nos processos contínuos são temperatura, vazão e pressão."[6]

\textbf{INSERIR FIGURA 2}

``Outra classificação aceita para os sistemas automatizados de produção está
relacionada ao grau de flexibilidade, sendo definidos três tipos básicos: auto-
mação rígida, programável e flexível. A posição relativa dos três tipos de
automação para os diferentes volumes e variedades dos produtos é mostrada
na Figura 3."[6]

\textbf{INSERIRR FIGURA 3}

``
\begin{itemize}
	\item \textbf{Automação rígida} – está baseada em uma linha de produção projetada
para a fabricação de um produto específico. Apresenta altas taxas de
produção e inflexibilidade do equipamento na acomodação da variedade
de produção.
	\item \textbf{Automação programável} – o equipamento de produção é projetado com
a capacidade de modificar a sequência de operações de modo a acomodar
diferentes configurações de produtos, sendo controlado por um programa
que é interpretado pelo sistema. Diferentes programas podem ser utilizados
para fabricar novos produtos. Esse tipo de automação é utilizado quando
o volume de produção de cada item é baixo.
	\item \textbf{Automação flexível} – reúne algumas das características da automação
rígida e outras da automação programável. O equipamento deve ser progra-
mado para produzir uma variedade de produtos com algumas características
ou configurações diferentes, mas a variedade dessas características é nor-
malmente mais limitada que aquela permitida pela automação programável.
\end{itemize}
"[6]

\textbf{VANTAGENS DA AUTOMAÇÃO APRESENTADAS PELO AUTOR DE [1]}

``
\begin{enumerate}
	\item Houve uma revolução industrial com automação de processos de manufatura.
	\item Automação é o uso da potência elétrica ou mecânica controlada por um
				sistema de controle inteligente (geralmente eletrônico) para aumentar
				a produtividade e diminuir os custos.
	\item A falta de automação pode aumentar o desemprego.
	\item Automação é um meio para aumentar a produtividade.
	\item A habilidade de controlar os passos de um processo é a chave da automação.
	\item Avanços na eletrônica tornaram possível o controle de sistemas complexos,
				a um baixo custo.
\end{enumerate} 
"[1]

\textbf{VANTAGENS DA AUTOMAÇÃO ARERSENTADAS PELO AUTOR DE [6]}

``Algumas razões que justificam a automação da produção e da manufatura
são as seguintes: aumento da produtividade, redução dos custos do traba-
lho, minimização dos efeitos da falta de mão de obra qualificada, redução
ou eliminação das atividades manuais rotineiras, aumento da segurança do
trabalhador, melhoria na uniformidade do produto, realização de processos
que não podem ser executados manualmente."[6]

\subsection{Justificativa}
\subsection{Objetivo Geral}

Este trabalho tem o propósito de/pretende/visa/ mostrar a importância da 
ultilização de ferramentas de simulação para a programação de máquinas de
automação de processos industriais.

(citar a bancada Festo?)

\subsection{Objetivo Especifico}

\begin{itemize}
  \item Projetar uma linha de produção industrial
  \item Implementar o projeto utilizando PLC (?)
  \item Simular a implementação com software adequados(?)
  \item Utilizar a implementação na planta real
  \item Verificar ...
\end{itemize}

\section{Referencial Teórico}

\section{Metodologia do trabalho}

\section{Cronograma}

\section{Referencias Bibliográficas}

[1] Automação Industrial - Marco Antonio Ribeiro

[2] Controladores logicos programaveis - Frank D. Petruzella

[3] Automação Industrial na Prática - Frank Lamb

[4] Controladores Lógicos Programáveis - Claiton Moro Franchi

[5] A quarta revolução industrial - Klaus Schwab

[6] Automacao Industrial - Leandro Roggia \& Rodrigo Cardozo Fuentes

[7] Introdução à Automação - Mauro Spinola

[8] IEC 61131

[9] IEC 61131-3

\end{document}