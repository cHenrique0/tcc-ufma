\documentclass[a4paper, 12pt]{article}
\usepackage[top=2cm, bottom=2cm, left=2.5cm, right=2.5cm]{geometry}
\usepackage[utf8]{inputenc}
\usepackage{amsmath, amsfonts, amssymb}
\usepackage{float}
\usepackage{graphicx}
\usepackage[brazil]{babel}
\usepackage{indentfirst}

\begin{document}

\begin{center}
\part*{Resumo}
\end{center}

\begin{center}
\part*{Abstract}
\end{center}

\section{Introdução}
1. Conceitos de Automação industrial

``Automação é a substituição do trabalho humano ou animal por máquina.
Automação é a operação de máquina ou de sistema automaticamente ou por
controle remoto, com a mínima interferência do operador humano.
Automação é o controle de processos automáticos. Automático significa ter
um mecanismo de atuação própria, que faça uma ação requerida em tempo 
determinado ou em resposta a certas condições."[1]

``O conceito de automação inclui a idéia de usar a potência elétrica ou 
mecânica para acionar algum tipo de máquina. Deve acrescentar à máquina
algum tipo de inteligência para que ela execute sua tarefa de modo mais
eficiente e com vantagens econômicas e de segurança."[1]

``Como vantagens, a máquina: Não reclama, não entra em greve, não pede 
aumento de salário, não precisa de férias, não requer mordomias.
Como nada é perfeito, a máquina tem as seguintes limitações: capacidade 
limitada de tomar decisões, deve ser programada ou ajustada para controlar
sua operação nas condições especificadas, necessita de calibração periódica
para garantir sua exatidão nominal, requer manutenção eventual para assegurar
que sua precisão nominal não se degrade."[1]

1.1. Automação e mão de obra:

``Com o advento do circuito integrado(1960) e do microprocessador (1970), a
quantidade de inteligência que pode ser embutida em uma máquina a um custo
razoável se tornou enorme. O número de tarefas complexas que podem ser feitas
automaticamente cresceu várias vezes. Atualmente, pode-se dedicar ao computador
pessoal para fazer tarefas simples e complicadas, de modo econômico."[1]

``A automação pode reduzir a mão de obra empregada, porém ela também e
ainda requer operadores. Em vez de fazer a tarefa diretamente, o operador
controla a máquina que faz a tarefa. Assim, a dona de casa deve aprender a
carregar a máquina de lavar roupa ou louça e deve conhecer suas limitações.
Operar a máquina de lavar roupa pode inicialmente parecer mais difícil que
lavar a roupa diretamente. Do mesmo modo, o operador de uma furadeira 
automática na indústria automobilística deve ser treinado para usar
a máquina com controle numérico que faz o furo realmente. A linha de 
montagem com robôs requer operadores para monitorar o desempenho desses robôs.
Quem tira o dinheiro do caixa eletrônico, deve possuir um cartão apropriado,
decorar uma determinada senha e executar uma série de comandos no teclado 
ou tela de toque."[1]

``Muitas pessoas pensam e temem que a automação significa perda de empregos,
quando pode ocorrer o contrário. De fato, falta de automação coloca muita gente
para trabalhar. Porém, estas empresas não podem competir economicamente com
outras por causa de sua baixa produtividade devida à falta de automação
e por isso elas são forçadas a demitir gente ou mesmo encerrar suas atividades.
Assim, automação pode significar ganho e estabilidade do emprego, por causa do
aumento da produtividade, eficiência e economia."[1]

1.2. Automação e controle:

``A automação está intimamente ligada à instrumentação. Os diferentes
instrumentos são usados para realizar a automação.
Historicamente, o primeiro termo usado foi o de controle automático de processo.
Foram usados instrumentos com as funções de medir, transmitir, comparar e
atuar no processo, para se conseguir um produto desejado com pequena ou
nenhuma ajuda humana. Isto é controle automático.
Com o aumento da complexidade dos processos, tamanho das plantas, exigências de
produtividade, segurança e proteção do meio ambiente, além do controle automático
do processo, apareceu a necessidade de monitorar o controle automático."[1]

``Por isso, para o autor, principalmente para a preparação de seus cursos e
divisão de assuntos, tem-se o controle automático aplicado a processo contínuo,
com predominância de medição, controle PID (proporcional, integral e derivativo).
O sistema de controle aplicado é o Sistema Digital de Controle Distribuído (SDCD),
dedicado a grandes plantas ou o controlador single loop, para aplicações
simples e com poucas malhas. Tem-se a automação associada ao controle automático,
para fazer sua monitoração, incluindo as tarefas de alarme e intertravamento.
A automação é também aplicada a processos discretos e de batelada, onde há muita
operação lógica de ligar e desligar e o controle seqüencial.
O sistema de controle aplicado é o Controlador Lógico Programável (CLP)."[1]

1.3. Automação e eletrônica

``Na década de 1970, era clássica a comparação entre as instrumentações
eletrônica e pneumática. Hoje, às vésperas do ano 2000, há a predominância da
eletrônica microprocessada.
Os sensores que medem o valor ou estado de variáveis importantes em um
sistema de controle são as entradas do sistema, mas o coração do sistema é o
controlador eletrônico microprocessado. Muitos sistemas de automação só se
tornaram possíveis por causa dos recentes e grandes avanços na eletrônica. 
Sistemas de controle que não eram práticos por causa de custo há cinco anos 
atrás hoje se tornam obsoletos por causa do rápido avanço da tecnologia."[1]

``A chave do sucesso da automação é o uso da eletrônica microprocessada que
pode fornecer sistemas eletrônicos programáveis. Por exemplo, a indústria
aeronáutica constrói seus aviões comerciais em uma linha de montagem,
mas personaliza o interior da cabine através de simples troca de um programa
de computador. A indústria automobilística usa robôs para soldar pontos e 
fazer furos na estrutura do carro. A posição dos pontos de solda, o diâmetro
e a profundidade dos furos e todas as outras especificações podem ser alteradas
através da simples mudança do programa do computador. Como o programa do 
computador é armazenado em um chip de memória, a alteração de linhas do programa
neste chip pode requerer somente alguns minutos. Mesmo quando se tem que 
reescrever o programa, o tempo e custo envolvidos são muitas vezes menores que o
tempo e custo para alterar as ferramentas."[1]

3. Sistemas de Automação

``A aplicação de automação eletrônica nos processos industriais resultou em
vários tipos de sistemas, que podem ser geralmente classificados como:
\begin{enumerate}
	\item Máquinas com controle numérico
	\item Controlador lógico programável
	\item Sistema automático de armazenagem e recuperação
	\item Robótica
	\item Sistemas flexíveis de manufatura.
\end{enumerate}

3.1 Maquinas com controle numerico

EXPLICAR RAPIDAMENTE

3.2 Controlador lógico programável

O controlador lógico programável é um equipamento eletrônico, digital, 
microprocessado, que pode:
\begin{enumerate}
	\item controlar um processo ou uma máquina
	\item ser programado ou reprogramado rapidamente e quando necessário
	\item ter memória para guardar o programa.
\end{enumerate}
O programa é inserido no controlador através de microcomputador, teclado
numérico portátil ou programador dedicado. O controlador lógico programável
varia na complexidade da operação que eles podem controlar, mas eles podem ser
interfaceados com microcomputador e operados como um DNC, para aumentar
sua flexibilidade. Por outro lado, eles são relativamente baratos, fáceis 
de projetar e instalar.

3.3. Sistema de armazenagem e
recuperação de dados

EXPLICAR RAPIDAMENTE

3.4. Robótica

EXPLICAR RAPIDAMENTE

3.5. Sistema de manufatura flexível

EXPLICAR RAPIDAMENTE
"[1]

VANTAGENS DA AUTOMAÇÃO APRESENTADAS PELO AUTOR

``
\begin{enumerate}
	\item Houve uma revolução industrial com automação de processos de manufatura.
	\item Automação é o uso da potência elétrica ou mecânica controlada por um
				sistema de controle inteligente (geralmente eletrônico) para aumentar
				a produtividade e diminuir os custos.
	\item A falta de automação pode aumentar o desemprego.
	\item Automação é um meio para aumentar a produtividade.
	\item A habilidade de controlar os passos de um processo é a chave da automação.
	\item Avanços na eletrônica tornaram possível o controle de sistemas complexos,
				a um baixo custo.
\end{enumerate} 
"[1]

\subsection{Justificativa}
\subsection{Objetivo Geral}
\subsection{Objetivo Especifico}

\section{Referencial Teórico}
As subseções são as tecnologias(?????????)

\section{Metodologia do trabalho}

\section{Cronograma}

\section{Referencias Bibliográficas}

\end{document}